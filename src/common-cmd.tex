\makeatletter
% 環境
\renewcommand{\(}{\hfill$\begin{aligned}[t]}
\renewcommand{\)}{\end{aligned}$\hfill\null}
\newenvironment{eqalign}{\[\begin{aligned}}{\end{aligned}\]}
\newenvironment{eqgather}{\[\begin{gathered}}{\end{gathered}\]}

% テキスト書体コマンド
\newcommand{\gs}{\sffamily}
\newcommand{\gsb}{\gs\bfseries}
\newcommand{\xrm}[1]{\textrm{#1}}
\newcommand{\xit}[1]{\textit{#1}}
\newcommand{\xsf}[1]{\textsf{#1}}
\newcommand{\xtt}[1]{\texttt{#1}}
\newcommand{\xbf}[1]{\textbf{#1}}
% 数式書体コマンド
\newcommand{\up}[1]{\symup{#1}} % 1文字セリフ・ローマン体
\newcommand{\mup}[1]{\mathrm{#1}} % 複数文字セリフ・ローマン体
\newcommand{\yit}[1]{\symit{#1}} % 1文字セリフ・イタリック体
% \newcommand{}[1]{\mathit{#1}} % 複数文字セリフ・イタリック体
\newcommand{\sfu}[1]{\symsfup{#1}} % サンセリフ・ローマン体
\newcommand{\sfi}[1]{\symsfit{#1}} % サンセリフ・イタリック体
\newcommand{\frk}[1]{\symfrak{#1}} % フラクトゥール
\newcommand{\bb}[1]{\symbb{#1}} % 黒板太字
\newcommand{\bbi}[1]{\symbbit{#1}} % 黒板太字イタリック体
\newcommand{\ncl}[1]{\symcal{#1}} % 新チャンセリー筆記体
\newcommand{\scr}[1]{\symscr{#1}} % 新チャンセリー筆記体・小文字,ラウンド筆記体
% - - - - - - -
\newcommand{\bup}[1]{\symbfup{#1}} % 太字セリフ・ローマン体
\newcommand{\bit}[1]{\symbfit{#1}} % 太字セリフ・イタリック体
\newcommand{\bsu}[1]{\symbfsfup{#1}} % 太字サンセリフ・ローマン体
\newcommand{\bsi}[1]{\symbfsfit{#1}} % 太字サンセリフ・イタリック体
\newcommand{\bfr}[1]{\symbffrak{#1}} % 太字フラクトゥール
% \newcommand{\bcl}[1]{??} % 太字旧チャンセリー筆記体
\newcommand{\bncl}[1]{\symbfcal{#1}} % 太字新チャンセリー筆記体
\newcommand{\bsc}[1]{\symbfscr{#1}} % 太字新チャンセリー筆記体・小文字,太字ラウンド筆記体
% ====== 筆記体メモ ======
% \cl P : 旧チャンセリー筆記体
% \ncl P = \symcal P : 新チャンセリー筆記体・大文字
% \scr p = \symscr p, \ncl p = \symcal p : 新チャンセリー筆記体・小文字
% \scr P = \symscr P : ラウンド筆記体
% \bcl P : 太字旧チャンセリー筆記体(未定義)
% \bncl P = \symbfcal P : 太字新チャンセリー筆記体・大文字
% \bsc p = \symbfscr p, \bncl p = \symbfcal p : 太字新チャンセリー筆記体・小文字
% \bsc P = \symbfscr P : 太字ラウンド筆記体

\newcommand{\opr}[1]{\operatorname{\sfu{#1}}}
\newcommand{\MO}[1]{\mathord{#1}}

% 雑記号
\newcommand{\para}{\textcolor{mygray}{$\linefeed$}\par}
\newcommand{\?}{\varstar}
\newcommand{\AND}{\hspace{.5em} / \hspace{.5em}}
\newcommand{\:}{\colon}
\newcommand{\…}{\dots}


% 両側デリミタ
\newcommand{\pr}[1]{\lleft(#1\rright)}
\newcommand{\cb}[1]{\lleft\{#1\rright\}}
\newcommand{\ab}[1]{\lleft\langle#1\rright\rangle}
\newcommand{\⟨}{\lleft\langle}
\newcommand{\⟩}{\rright\rangle}


% 片側デリミタ
\makeatletter
\newcommand{\cuc}[1]{\begin{cases}#1\end{cases}} % {
\newenvironment{sqcases}{\matrix@check\sqcases\env@sqcases}{\endarray\right.}\def\env@sqcases{\let\@ifnextchar\new@ifnextchar\left\lbrack\def\arraystretch{1.2}\array{@{}l@{\quad}l@{}}}
\newcommand{\sqc}[1]{\begin{sqcases}#1\end{sqcases}} % [
\makeatother

% 論理記号タイプ
\newcommand{\ra}{\varrightarrow}
\newcommand{\lra}{\varleftrightarrow}
\newcommand{\Defn}{\mathrel{:⇔}}
\newcommand{\defn}{\mathrel{:\lra}}

% 集合
\newcommand{\cls}[2]{\cb{#1 \mmrel| #2}}
\newcommand{\dom}[1]{\opr{dom}\pr{#1}}
\NewDocumentCommand{\funcs}{mmO{}}{\leftidx{^{#1}}{#2}{#3}}
\newcommand{\id}{\mup{id}}

\makeatother