\makeatletter

\def\mydoller{$}\def\mydoller{$}
% 環境
\ExplSyntaxOn
\bool_set_true:N \g_my_OutsideEq_bool
\tl_new:N \g_my_EqType_tl
\NewDocumentCommand { \? } { O{} } {
  \bool_if:NTF \g_my_OutsideEq_bool {
    \tl_if_empty:nTF { #1 } {
      \csname [\endcsname
      \tl_gset:Nn \g_my_EqType_tl { e }
    } {
      \tl_if_eq:nnTF { #1 } { t } {
        \hfill\mydoller
        \begin {aligned}[t]
        \tl_gset:Nn \g_my_EqType_tl { t }
      }
      \tl_if_eq:nnTF { #1 } { a } {
        \csname [\endcsname
        \begin {aligned}
        \tl_gset:Nn \g_my_EqType_tl { a }
      } {}
      \tl_if_eq:nnTF { #1 } { g } {
        \csname [\endcsname
        \begin {gathered}
        \tl_gset:Nn \g_my_EqType_tl { g }
      } {}
    }
    \bool_set_false:N \g_my_OutsideEq_bool
  } {
    \tl_if_eq:NnTF \g_my_EqType_tl { e } {
      \csname ]\endcsname
    } {}
    \tl_if_eq:NnTF \g_my_EqType_tl { t } {
      \end {aligned}
      \mydoller\hfill\null
    } {}
    \tl_if_eq:NnTF \g_my_EqType_tl { a } {
      \end {aligned}
      \csname ]\endcsname
    } {}
    \tl_if_eq:NnTF \g_my_EqType_tl { g } {
      \end {gathered}
      \csname ]\endcsname
    } {}
    \bool_set_true:N \g_my_OutsideEq_bool
  }
}
\ExplSyntaxOff


% テキスト書体コマンド
\newcommand{\gs}{\sffamily}
\newcommand{\gsb}{\gs\bfseries}
\newcommand{\xrm}[1]{\textrm{#1}}
\newcommand{\xit}[1]{\textit{#1}}
\newcommand{\xsf}[1]{\textsf{#1}}
\newcommand{\xtt}[1]{\texttt{#1}}
\newcommand{\xbf}[1]{\textbf{#1}}
% 数式書体コマンド
\newcommand{\up}[1]{\symup{#1}} % 1文字セリフ・ローマン体
\newcommand{\mup}[1]{\mathrm{#1}} % 複数文字セリフ・ローマン体
\newcommand{\yit}[1]{\symit{#1}} % 1文字セリフ・イタリック体
% \newcommand{}[1]{\mathit{#1}} % 複数文字セリフ・イタリック体
\newcommand{\sfu}[1]{\symsfup{#1}} % サンセリフ・ローマン体
\newcommand{\sfi}[1]{\symsfit{#1}} % サンセリフ・イタリック体
\newcommand{\frk}[1]{\symfrak{#1}} % フラクトゥール
\newcommand{\bb}[1]{\symbb{#1}} % 黒板太字
\newcommand{\bbi}[1]{\symbbit{#1}} % 黒板太字イタリック体
\newcommand{\ncl}[1]{\symcal{#1}} % 新チャンセリー筆記体
\newcommand{\scr}[1]{\symscr{#1}} % 新チャンセリー筆記体・小文字,ラウンド筆記体
% - - - - - - -
\newcommand{\bup}[1]{\symbfup{#1}} % 太字セリフ・ローマン体
\newcommand{\bit}[1]{\symbfit{#1}} % 太字セリフ・イタリック体
\newcommand{\bsu}[1]{\symbfsfup{#1}} % 太字サンセリフ・ローマン体
\newcommand{\bsi}[1]{\symbfsfit{#1}} % 太字サンセリフ・イタリック体
\newcommand{\bfr}[1]{\symbffrak{#1}} % 太字フラクトゥール
% \newcommand{\bcl}[1]{??} % 太字旧チャンセリー筆記体
\newcommand{\bncl}[1]{\symbfcal{#1}} % 太字新チャンセリー筆記体
\newcommand{\bsc}[1]{\symbfscr{#1}} % 太字新チャンセリー筆記体・小文字,太字ラウンド筆記体
% ====== 筆記体メモ ======
% \cl P : 旧チャンセリー筆記体
% \ncl P = \symcal P : 新チャンセリー筆記体・大文字
% \scr p = \symscr p, \ncl p = \symcal p : 新チャンセリー筆記体・小文字
% \scr P = \symscr P : ラウンド筆記体
% \bcl P : 太字旧チャンセリー筆記体(未定義)
% \bncl P = \symbfcal P : 太字新チャンセリー筆記体・大文字
% \bsc p = \symbfscr p, \bncl p = \symbfcal p : 太字新チャンセリー筆記体・小文字
% \bsc P = \symbfscr P : 太字ラウンド筆記体

% 非記号系コマンド
\newcommand{\rx}[1]{{\color{myred}#1}}
\newcommand{\Release}[1]{\rx{Release (#1)}}
\newcommand{\myTab}{\makebox[2em][l]{\scriptsize \color{mywhitegray}$⇥$}}
% \ExplSyntaxOn
% \int_step_inline:nnn { 2 } { 10 } {
%   \expandafter \newcommand
%   \csname \prg_replicate:nn { #1 } { T } \endcsname
%   {\prg_replicate:nn { #1 } { \T }}
% }
% \ExplSyntaxOff
\newcommand{\Show}{\textbf{Show: }}
\newcommand{\Assume}[1]{\textbf{Assume: }{\color{myred}#1}}
\ExplSyntaxOn
\keys_define:nn { step } {
  label .tl_set:N = \l_step_label_tl,
  tab .int_set:N = \l_step_tab_int,
  show .bool_set:N = \l_step_show_bool,
  show .default:n = true,
  assume .bool_set:N = \l_step_assume_bool,
  assume .default:n = true,
  fml .tl_set:N = \l_step_fml_tl,
  by .clist_set:N = \l_step_by_clist,
}
\NewDocumentCommand { \step } { m } {
  \item
  \group_begin:
    \keys_set:nn { step } { #1 }
    \tl_if_empty:NTF \l_step_label_tl {} { \label{\l_step_label_tl} }
    \prg_replicate:nn { \l_step_tab_int } { \myTab }
    \bool_if:NTF \l_step_show_bool { \Show } {}
    \bool_if:NTF \l_step_assume_bool { \Assume{\l_step_fml_tl} } { \l_step_fml_tl }
    \clist_if_empty:NTF \l_step_by_clist {} {
      \seq_set_from_clist:NN \l_tmpa_seq \l_step_by_clist
      \seq_set_map:NNn \l_tmpa_seq \l_tmpa_seq {
        \tl_set:Nn \l_tmpa_tl { ##1 }
        \tl_if_in:NnTF \l_tmpa_tl { Axm } {
          \nameref{\l_tmpa_tl}
        }{
          \tl_if_in:NnTF \l_tmpa_tl { Dfn } {
            \cref{\l_tmpa_tl}
          }{
            \tl_if_in:NnTF \l_tmpa_tl { Thm } {
              \cref{\l_tmpa_tl}
            }{
              \ref{\l_tmpa_tl}
            }
          }
        }
      }
      \hspace{.5em} \hrulefill \hspace{.5em} (
        \seq_use:Nn \l_tmpa_seq { ,\ }
      )    
    }
  \group_end:
}
\ExplSyntaxOff


% 記号タイプ
\def\opr{\@ifstar\@opr\@@opr}
\def\@opr#1{\operatorname{#1}}
\def\@@opr#1{\operatorname*{#1}}
\newcommand{\Opr}[1]{\operatorname{\sfu{#1}}}
\newcommand{\MO}[1]{\mathord{#1}}

% 雑記号
\newcommand{\para}{\textcolor{mygray}{$\linefeed$}\par}
\newcommand{\:}{\colon}
\newcommand{\…}{\dots}
\newcommand{\otw}{\text{Otherwise}}


% 両側デリミタ
\newcommand{\pr}[1]{\lleft(#1\rright)}
\newcommand{\cb}[1]{\lleft\{#1\rright\}}
\newcommand{\ab}[1]{\lleft\langle#1\rright\rangle}
\newcommand{\⟨}{\lleft\langle}
\newcommand{\⟩}{\rright\rangle}


% 片側デリミタ
\makeatletter
\newcommand{\cuc}[1]{\begin{cases}#1\end{cases}} % {
\newenvironment{sqcases}{\matrix@check\sqcases\env@sqcases}{\endarray\right.}\def\env@sqcases{\let\@ifnextchar\new@ifnextchar\left\lbrack\def\arraystretch{1.2}\array{@{}l@{\quad}l@{}}}
\newcommand{\sqc}[1]{\begin{sqcases}#1\end{sqcases}} % [
\makeatother

% 論理記号タイプ
\newcommand{\ra}{\varrightarrow}
\newcommand{\la}{\varleftarrow}
\newcommand{\lra}{\varleftrightarrow}
\newcommand{\Defn}{\mathrel{:⇔}}
\newcommand{\defn}{\mathrel{:\lra}}

% 集合
\newcommand{\cls}[2]{\cb{#1 \mmrel| #2}}
\newcommand{\dom}[1]{\opr{dom} \pr{#1}}
\newcommand{\img}[1]{\opr{img} \pr{#1}}
\NewDocumentCommand{\funcs}{mmO{}}{\leftidx{^{#1}}{#2}{#3}}
\newcommand{\id}{\mup{id}}

\makeatother
